Automation is preferred in most areas nowadays. It saves both time and cost. In the
current education system, answer scripts are evaluated by hand and this is time consuming. The
result declaration is in turn delayed. Handwriting recognition is needed along with semantic
analysis of text for the automation process. The technologies to be used for this are Deep
Learning and Natural Language Processing (NLP). 

Handwriting recognition of different languages are challenging issues and receive 
extensive attention from researchers.
In recent years, numerous handwritten datasets have been
published in the field to promote the advancement of the
community. In general, handwritten datasets can be divided
into two categories, i.e., online and offline datasets. For
example, there are offline handwritten datasets such as
French paragraph dataset Rimes \cite{augustin2006rimes}, English text dataset
IAM \cite{marti2002iam}, Arabic datasets of IFN/ENIT \cite{pechwitz2002baseline} and KHATT \cite{mahmoud2012khatt},
Chinese dataset CASIA-HWDB \cite{liu2011casia} and HIT-MW \cite{su2006hit}. For
online handwritten datasets, there are Japanese text datasets
Kondate \cite{matsushita2014database} and character dataset TUAT Nakayosi\_t and
Kuchibue\_d \cite{nakagawa2004collection}, English text dataset IAM-OnDB \cite{liwicki2005iam},
Chinese datasets SCUT-COUCH2009 \cite{jin2011scut}, CASIAOLHWDB \cite{liu2011casia}, and ICDAR2013 competition set \cite{yin2013icdar}.
In this project, we use the IAM dataset for training our LSTM model.

Generally in an answer paper evaluation, a human grader assesses and assigns
a score to a submission which is written concerning
an answer's prompt. This is a laborious and tiring task for
the graders. Also, human graders can be imperfect; they are
susceptible to biases and irregularities based on other chores
and activities they do in life \cite{shermis2003automated}. Different human graders also
have different grading styles and can also tend to give an overall higher grade just 
based on one good impression regarding
a particular aspect. A computer system can overcome all
these human shortcomings by uniform assessment throughout. 
Understanding human language is considered a laborious
task due to its complexity. There are numerous ways to
arrange words in a sentence. Also, words can have multiple meanings in different contexts. 
Therefore context-based
knowledge is necessary to decipher the sentences correctly.

Our proposed system makes use of an RNN architecture model like LSTM to recognize
handwritten answer sheets and convert them to text. This text is semantically 
analyzed using an NLP model with an answer key to generate results. 