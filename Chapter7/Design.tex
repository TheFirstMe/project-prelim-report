\section{Flowcharts}
\begin{figure}[!h]
    \vspace{1cm}
        \hspace{-2cm}
        \begin{tikzpicture}[node distance=2.5cm,thick,scale=0.8, every node/.style={scale=0.8}]
            \node (start) [startstop] {Start};
            \node (pro1) [process, below of=start, text width=\textwidth-10cm] {Upload answer paper \& answer key};
            \node (pro2) [process, below of=pro1] {Extract student details};
            \node (dec1) [decision, right of=pro2, xshift=3cm] {Student exists?};
            \node (pro3) [process, right of=dec1, xshift=3cm] {Extract answers};
            \node (pro4) [process, text width=\textwidth-10cm, below of=pro3, yshift=-1cm] {Map answers to answer keys};
            \node (pro5) [process, right of=pro4, xshift=3cm] {Evaluate answers};
            \node (pro6) [process, above of=pro5, yshift=1cm] {Generate report};
            \node (pro7) [process, above of=pro6] {Send report to student};
            \node (end) [startstop, above of=pro7] {End};
            \node (error) [process, left of=end, xshift=-3cm] {Error};

            \draw [arrow] (start) -- (pro1);
            \draw [arrow] (pro1) -- (pro2);
            \draw [arrow] (pro2) -- (dec1);
            \draw [arrow] (dec1) -- node[anchor=south] {yes} (pro3);
            \draw [arrow] (dec1) |- node[anchor=south west] {no} (error);
            \draw [arrow] (pro3) -- (pro4);
            \draw [arrow] (pro4) -- (pro5);
            \draw [arrow] (pro5) -- (pro6);
            \draw [arrow] (pro6) -- (pro7);
            \draw [arrow] (pro7) -- (end);
            \draw [arrow] (error) -- (end);
        \end{tikzpicture}
        \vspace{0.5cm}
        \caption{Flowchart}
        \label{fig:flowchart-1}
    \end{figure}
\noindent The flowchart of our system is shown in Figure \ref{fig:flowchart-1} which
represents the work flow. First, the teacher uploads the answer paper and the answer key to
the system. The handwriting recognition system extracts the students details from the front sheet
of the answer booklet and checks if the student record exists in the database. If no such
student exists, there is no further reason to extract the answers from the rest of the answer booklet.
So an error is raised. If the student exists, then the answers are extracted and mapped to
each question provided in the answer key.

The resultant answer is then provided to the NLP model where semantic analysis is done. Based
on the similarity, grades are given and the result is generated and stored in a database. 
Based on the student details extracted previously, the generated results are sent to individual
students.
\section{Data Flow Diagrams}
\subsection{Level-0 Data Flow Diagram}
\begin{figure}[!htb]
    \centering
    \begin{tikzpicture}[->,>=stealth',auto,node distance=3cm,thick,scale=0.8, every node/.style={scale=0.8}]
        \node (proc1) [dfdprocess] {Answer paper evaluation};
        \node (teacher) [dfdentity, left of=proc1, xshift=-3cm] {Teacher};
        \node (student) [dfdentity, right of=proc1, xshift=3cm] {Student};

        % \draw [arrow] (teacher) -- node[anchor=south, text width=3.9cm, text centered] {Answer paper \& Answer key} (proc1);
        \draw [arrow] (proc1) -- node[anchor=south] {Result} (student);

        \path
            (teacher) edge[bend right] node [text centered, anchor=north] {Answer paper} (proc1);
        \path    
            (teacher) edge[bend left] node [text centered] {Answer key} (proc1);
    \end{tikzpicture}
    \vspace{0.5cm}
    \caption{Data Flow Diagram Level 0}
    \label{fig:dfd-0}
\end{figure}
\noindent Figure \ref{fig:dfd-0} shows the Level-0 Data Flow Diagram of our system. 
As every level 0
DFD is designed, this diagram also gives the abstract view of our system. The external entities
are the \emph{Teacher} and \emph{Student}. Our system is shown as a single process \emph{Answer
paper evaluation}. The \emph{Teacher} uploads the answer paper and answer key to the \emph{
Answer paper evaluation} process which gives out the result to the \emph{Student}.
\subsection{Level-1 Data Flow Diagram}
\begin{figure}[!htb]
    \centering
    \begin{tikzpicture}[node distance=3cm,thick,scale=0.8, every node/.style={scale=0.8}]
        \node (proc1) [dfdprocess] {Handwriting recognition};
        \node (teacher) [dfdentity, left of=proc1, xshift=-3cm] {Teacher};
        \node (proc2) [dfdprocess, below of=teacher, yshift=-2cm] {Question to Answer key mapping};
        \node (proc3) [dfdprocess, right of=proc2, xshift=3cm] {Semantic evaluation};
        \node (result) [data store, right of=proc1, xshift=4cm] {Generate result};
        \node (student) [dfdentity, right of=proc3, xshift=4cm] {Student};

        \draw [arrow] (teacher) -- node[anchor=south] {Answer paper} (proc1);
        \draw [arrow] (teacher) -- node[rotate=-90, anchor=south] {Answer key} (proc2);
        \draw[arrow] (proc1) -- node[rotate=40, anchor=south] {Answers} (proc2);
        \draw[arrow] (proc2) -- node[anchor=south] {Answers} (proc3);
        \draw[arrow] (proc3) -- node[rotate=36,anchor=south] {Evaluated Answers} (result);
        \draw[arrow] (result) -- node[rotate=-90, anchor=south] {Result} (student);
        \draw[arrow] (proc1) -- node[anchor=south] {Student details} (result);
    \end{tikzpicture}
    \vspace{0.5cm}
    \caption{Data Flow Diagram Level 1}
    \label{fig:dfd-1}
\end{figure}
\noindent Figure \ref{fig:dfd-1} shows the Level-1 Data Flow Diagram of our system.
It is a more detailed representation of the Level-0 Data Flow Diagram.
The external entities are same as before: \emph{Teacher} and \emph{Student}.
The \emph{Answer paper evaluation} process is broken down into several parts.
Now, the answer paper from the \emph{Teacher} goes to a \emph{Handwriting recognition}
process. The answer key from the \emph{Teacher}
and the recognized answer from the 
\emph{Handwriting recognition} goes to a \emph{Question to Answer key mapping} process.
The modified answers from the \emph{Question to Answer key mapping} process
goes to the \emph{Semantic evaluation} process. Student details from
the \emph{Handwriting recognition} process and evaluated answers
from the \emph{Semantic evaluation} process goes to the
\emph{Generate result} data store where the result is stored 
and also result to that particular student is sent.
