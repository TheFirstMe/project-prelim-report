\begin{figure}[!htb]
    \hspace{-1cm}
    \begin{tikzpicture}[node distance=3cm,thick,scale=0.9, every node/.style={scale=0.9}]
        \node (proc1) [dfdprocess] {Handwriting recognition};
        \node (teacher) [dfdentity, left of=proc1, xshift=-3cm] {Teacher};
        \node (proc2) [dfdprocess, below of=teacher, yshift=-2cm] {Question to Answer key mapping};
        \node (proc3) [dfdprocess, right of=proc2, xshift=3cm] {Semantic evaluation};
        \node (result) [data store, right of=proc1, xshift=4cm] {Generate result};
        \node (student) [dfdentity, right of=proc3, xshift=4cm] {Student};

        \draw [arrow] (teacher) -- node[anchor=south] {Answer paper} (proc1);
        \draw [arrow] (teacher) -- node[rotate=-90, anchor=south] {Answer key} (proc2);
        \draw[arrow] (proc1) -- node[rotate=40, anchor=south] {Answers} (proc2);
        \draw[arrow] (proc2) -- node[anchor=south] {Answers} (proc3);
        \draw[arrow] (proc3) -- node[rotate=36,anchor=south] {Evaluated Answers} (result);
        \draw[arrow] (result) -- node[rotate=-90, anchor=south] {Result} (student);
        \draw[arrow] (proc1) -- node[anchor=south] {Student details} (result);
    \end{tikzpicture}
    \vspace{0.5cm}
    \caption{Data Flow Diagram Level 1}
    \label{fig:dfd-1}
\end{figure}