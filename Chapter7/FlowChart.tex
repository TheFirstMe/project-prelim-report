\begin{figure}[!h]
    \vspace{1cm}
        \hspace{-2cm}
        \begin{tikzpicture}[node distance=2.5cm,thick,scale=0.8, every node/.style={scale=0.8}]
            \node (start) [startstop] {Start};
            \node (pro1) [process, below of=start, text width=\textwidth-10cm] {Upload answer paper \& answer key};
            \node (pro2) [process, below of=pro1] {Extract student details};
            \node (dec1) [decision, right of=pro2, xshift=3cm] {Student exists?};
            \node (pro3) [process, right of=dec1, xshift=3cm] {Extract answers};
            \node (pro4) [process, text width=\textwidth-10cm, below of=pro3, yshift=-1cm] {Map answers to answer keys};
            \node (pro5) [process, right of=pro4, xshift=3cm] {Evaluate answers};
            \node (pro6) [process, above of=pro5, yshift=1cm] {Generate report};
            \node (pro7) [process, above of=pro6] {Send report to student};
            \node (end) [startstop, above of=pro7] {End};
            \node (error) [process, left of=end, xshift=-3cm] {Error};

            \draw [arrow] (start) -- (pro1);
            \draw [arrow] (pro1) -- (pro2);
            \draw [arrow] (pro2) -- (dec1);
            \draw [arrow] (dec1) -- node[anchor=south] {yes} (pro3);
            \draw [arrow] (dec1) |- node[anchor=south west] {no} (error);
            \draw [arrow] (pro3) -- (pro4);
            \draw [arrow] (pro4) -- (pro5);
            \draw [arrow] (pro5) -- (pro6);
            \draw [arrow] (pro6) -- (pro7);
            \draw [arrow] (pro7) -- (end);
            \draw [arrow] (error) -- (end);
        \end{tikzpicture}
        \vspace{0.5cm}
        \caption{Flowchart}
        \label{fig:flowchart-1}
    \end{figure}