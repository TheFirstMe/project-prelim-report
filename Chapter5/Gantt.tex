
\begin{figure}[!htb]
    \centering
    \noindent\resizebox{\textwidth}{!}{
    
        \begin{ganttchart}[%Specs
            x unit = 2cm,  %<---------------------- New x unit 
            y unit title=0.8cm,
            y unit chart=1cm,
            vgrid, hgrid,
            title height=1,
        %     title/.style={fill=none},
            title label font=\bfseries\footnotesize,
            bar/.style={fill=black},
            bar height=0.5,
            bar left shift=0.05,
            bar right shift=-0.05,
            progress label text={},
            group right shift=0,
            group top shift=0.7,
            group height=.3,
            group peaks width={0.2},
            inline]{1}{7}
           %labels
        
           \gantttitle{2019}{3}
           \gantttitle{2020}{4} \\                 % title 
           \gantttitle{October}{1}                      % title 3
           \gantttitle{November}{1}
           \gantttitle{December}{1}
           \gantttitle{January}{1}
           \gantttitle{February}{1}
           \gantttitle{March}{1}
           \gantttitle{April}{1}\\
        
           % Setting group if any
        
        %    \ganttgroup[inline=false]{Group 1}{1}{5}\\ 
        
           \ganttbar[inline=false]{Literature Survey}{1}{2}\\
           \ganttbar[progress=0,inline=false]{Create \& train RNN model}{3}{4}\\       
           \ganttbar[progress=0,inline=false]{Separate answers \& map to answer key}{4}{4}\\
           \ganttbar[progress=0,inline=false]{Create \& train NLP Model}{4}{5}\\
           \ganttbar[progress=0,inline=false]{Test Primary System}{6}{6}\\
           \ganttbar[progress=0,inline=false]{Evaluation of performance}{7}{7}
        \end{ganttchart}
        }
        \vspace{0.5cm}
        \caption{Gantt Chart}
        \label{fig:gantt}
\end{figure}