Handwriting recognition is needed to digitize answer papers. Deep learning is preferred
for handwriting recognition for better accuracy. Zhu \emph{et al.} \cite{8565866} proposed a methodology for
offline text recognition in answer papers. They noticed that most existing studies and public
datasets for handwritten Chinese text recognition are based on the regular documents with clean
and blank background, lacking research reports for handwritten text recognition on challenging
areas such as educational documents and financial bills. They focused on examination paper
text recognition and construct a challenging dataset named examination paper text (SCUT-EPT)
dataset, which contains 50 000 text line images (40,000 for training and 10,000 for testing)
selected from the examination papers of 2,986 volunteers. The proposed SCUT-EPT dataset
presents numerous novel challenges, including character erasure, text line supplement,
character/phrase switching, noised background, nonuniform word size, and unbalanced text
length. In their experiments, the current advanced text recognition methods, such as
convolutional recurrent neural network (CRNN) exhibits poor performance on the proposed
SCUT-EPT dataset, proving the challenge and significance of the dataset. Nevertheless,
through visualizing and error analysis, they observed that humans can avoid vast majority of
the error predictions, which reveal the limitations and drawbacks of the current methods for
handwritten Chinese text recognition (HCTR). Finally, three popular sequence transcription
methods, connectionist temporal classification (CTC), attention mechanism, and cascaded 
attention-CTC were investigated for HCTR problem. Although the attention mechanism has
been proved to be very effective in English scene text recognition, its performance is far inferior
to the CTC method in the case of HCTR with large-scale character set.


Manual grading of essays by humans is time-consuming and likely to be susceptible to
inconsistencies and inaccuracies. In recent years, an abundance of research has been done to
automate essay evaluation processes, yet little has been done to take into consideration the
syntax, semantic coherence and sentiments of the essay's text together. Janda \emph{et al.} \cite{8788526} proposed
a methodology for automated essay evaluation. The proposed system incorporates not just the
rule-based grammar and surface level coherence check but also includes the semantic similarity
of the sentences. They proposed to use Graph-based relationships within the essay's content and
polarity of opinion expressions. Semantic similarity is determined between each statement of
the essay to form these Graph-based spatial relationships and novel features are obtained from
it. Their algorithm uses 23 salient features with high predictive power, which is less than the
current systems while considering every aspect to cover the dimensions that a human grader
focuses on. Fewer features helps get rid of the redundancies of the data so that the predictions
are based on more representative features and are robust to noisy data. The resulting agreement
between human grader's score and the system's prediction is measured using Quadratic
Weighted Kappa (QWK). Their system produces a QWK of 0.793.
